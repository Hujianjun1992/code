\documentclass[9pt,a4paper]{article}

\XeTeXlinebreaklocale "zh"
\XeTeXlinebreakskip = 0pt plus 1pt minus 0.1pt
\usepackage[top=1in,bottom=1in,left=1in,right=1in]{geometry}
\usepackage{float}
\usepackage{fontspec}
\newfontfamily\zhfont[BoldFont=Adobe Heiti Std]{Adobe Song Std}
\newfontfamily\zhpunctfont{Adobe Song Std}
\setmainfont{Times New Roman}
\usepackage{indentfirst}
\usepackage{zhspacing}
\zhspacing

\usepackage{listings}
\lstset{
	frame=shadowbox,
	basicstyle=\footnotesize,
	numbers=left,
	numberstyle=\footnotesize,
	tabsize=2,
	breaklines=true,
	xleftmargin=2em,
	xrightmargin=2em,
	aboveskip=1em,
	showspaces=false
	}
	
\usepackage{framed}   
%%%% 段落首行缩进两个字 %%%%
\makeatletter
\let\@afterindentfalse\@afterindenttrue
\@afterindenttrue
\makeatother
\setlength{\parindent}{2em}  %中文缩进两个汉字位


%%%% 下面的命令重定义页面边距,使其符合中文刊物习惯 %%%%
\addtolength{\topmargin}{-54pt}
\setlength{\oddsidemargin}{0.63cm}  % 3.17cm - 1 inch
\setlength{\evensidemargin}{\oddsidemargin}
\setlength{\textwidth}{14.66cm}
\setlength{\textheight}{24.00cm}    % 24.62

%%%% 下面的命令设置行间距与段落间距 %%%%
\linespread{1.4}
% \setlength{\parskip}{1ex}
\setlength{\parskip}{0.5\baselineskip}


\begin{document}
	
%%%% 重定义 %%%%
\renewcommand{\contentsname}{目录}  % 将Contents改为目录
\renewcommand{\abstractname}{摘要}  % 将Abstract改为摘要
\renewcommand{\refname}{参考文献}   % 将References改为参考文献
\renewcommand{\indexname}{索引}
\renewcommand{\figurename}{图}
\renewcommand{\tablename}{表}
\renewcommand{\appendixname}{附录}
%\renewcommand{\algorithm}{算法}
	
	
%%%% 定义标题格式,包括title,author,affiliation,email等 %%%%
\title{RGB-D SLAM入门 \\ RGB-D SLAM Tutorial \\
	第二讲~~从图像到点云}
\author{高~翔\footnote{Email: gaoxiang12@mails.tsinghua.edu.cn}\\[2ex]
	 清华大学自动化系\\[2ex]
}
\maketitle

\section{读取图像}
小萝卜:师兄我又来了!今天说好了要读图了,你可别再偷懒了!

师兄:好好好,我们这就来读。

趁着你对cmake的知识还没忘,我们在

\begin{framed}
未完待续	
\end{framed}
	

\end{document}